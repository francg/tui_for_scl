Pervasive information visualisation is a fundamental tool to support learning in smart cities, for example to promote sustainable behaviors and social interaction. In this setting, information supporting learning goals is usually provided via screen-based interfaces such as public large displays and smart-phones. Those interfaces pose several usability issues and ignore the space of opportunities offered by research in tangible and embodied interaction.
Building on a review of existing works, we identify the main limitations of traditional approaches based on large displays and smart-phones, first from a technological point of view then connecting to implications for user interaction and experience.
Opportunities and challenges are then compared to the peculiarities of TUIs that can be useful and effective in smart city learning scenarios.
%Building on a review of existing works, we identify opportunities and challenges related to the development TUIs for smart city learning.
We explore new design opportunities allowed by TUIs in smart city learning scenarios of interest. We propose the use of \emph{Tiles}, an experimental prototyping platform, to quickly implement scenarios in working prototypes.
\emph{Tiles} allow to digitalize and augment regular objects, collect sensor data and handle user notification and interaction in several ways using onboard I/O components.
The architecture behind the electronic hardware is cloud-based. \emph{Tiles} functions are wrapped behind a simple, language-agnostic API, that allow both developers and designers to take advantage of the capabilities offered.
%We explore those design opportunities by providing smart city learning scenarios and we propose the use of \emph{Tiles}, an experimental prototyping platform, to quickly implement scenarios in working prototypes.  
