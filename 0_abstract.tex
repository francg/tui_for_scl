Pervasive information visualisation is a fundamental tool to support learning in smart cities, for example to promote sustainable behaviors and social interaction. In this setting, information supporting learning goals is usually provided via screen-based interfaces such as public large displays and smart-phones. Those interfaces pose several usability issues and ignore the space of opportunities offered by research in tangible and embodied interaction. Building on a review of existing works, we identify opportunities and challenges related to the development TUIs for smart city learning. We explore those design opportunities by providing smart city learning scenarios and we propose the use of \emph{Tiles}, an experimental prototyping platform, to quickly implement scenarios in working prototypes.  

