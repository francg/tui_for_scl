Research and innovation in the smart city is becoming a critical field for our society. Half of the world population is already living in cities, research in smart cities can have a big impact on a considerable fraction of world's population.
Studies demonstrate that social connections in cities stimulates creativity and improves work quality\cite{florida_cities_2005}. This is only one of the reasons why the percentage of people living in urban environments is growing.

Cities are important research playgrounds for substantial improvements in safety and quality of life.
Flow of information and people are easily observable in a highly connected urban environment, at the same time these information affect a high percentage of the humanity. A critical scenario emerges, where decision makers, citizens and researchers can shape the urban environment that influence our society.

It is also important to support the learning process within the urban communities: \textit{lifelong learning} can be used to promote sustainable behaviors and knowledge building in the city, with the ultimate goal of increasing community awareness of the urban space and increase quality of life.

Smart cities present, by definition, a strong technological component.
In Technology Enhanced Learning (TEL), the role of technology is to direct, foster thinking and facilitate the acquisition of higher order skills\cite{goodyear_technologyenhanced_2010}.
Current research applied to learning in the cities seem to focus on two main technological approaches: situated large displays and mobile devices, intended as tablets and smartphones\cite{luff_mobility_1998}.
%% add more ref
There seem to be space for additional research that empower more novel technological approaches like ubiquitous computing, internet of things (IoT), tangible interfaces and augmented objects. Some works adopted these technologies in the past\cite{stanton_classroom_2001}, but their applications were mainly oriented to support learning as it happen in conventional schools and classrooms.

Modern technologies and rapid prototyping toolkits can be of great help to overcome the limiting factors of more traditional technological approaches based on large screens and smartphones, especially when the learning environment can be as wide and heterogeneous as a city.

In currently published research there's little use of rich or unobtrusive forms of interactions between technologies and users.
Our approach is to research new forms of utilizing and collecting data in the city and at the same time experimenting with novel strategies of interaction between users and technology.
The potential of custom-designed hardware can open new possibilities in terms of communication efficiency, methods of interaction with the user and sensor data collection.
Glances, gesture based input and micro interactions are some of the approaches possible with the use of latest toolkits for IoT.
%% ref toolkit

Traditional technology is otherwise a limiting factor: mobile devices and large screens support a very strict and confined set of interaction strategies. It's often not possible to tailor the user experience to properly fit the specific scenario because technology is too limiting.
Our goal is to design aiming at the best possible strategy for the users, building the technology around this process and avoiding the constraints typically introduced by more general-purpose hardware/software combinations.

Some of the activities that are needed and connected to the smart city learning scenarios involves:

\begin{itemize}
\item data collection
\item data visualization
\item data processing
\item user interaction
\item sharing and cooperation of both electronics and informations
\end{itemize}
