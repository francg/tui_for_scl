\subsection{Connection with smart cities}

Some of the activities that are needed and connected to the smart city learning scenarios involves:

\begin{itemize}
\item data collection
\item data visualization
\item data processing
\item user interaction
\item sharing and cooperation of both electronics and informations
\end{itemize}

We identified how these activities can be supported with tangible interfaces.

\subsubsection{Data collection}
Sensor data can be captured embedding proper electronics in augmented objects or electronics toolkits.
Some of this \textit{sensing} capabilities are necessary to actually allow augmented user interactions (e.g. accelerometers, gyro).
Despite this, raw data sensed can be manipulated and processed also for other purposed more connected to the ambient sensing than remaining limited to only sense user (rich) interaction.
The opportunity to flexibly and quickly prototype custom devices, allow to embed with a relatively low cost several different sensors.
The opportunity to situate devices both on the environment and physically on the users widens the domain where data collection take place.

\subsubsection{Data visualization}
Citizens living everyday in the city can be subjected to common life situations where the most suitable strategies of consuming information can be fairly different.
TUIs can efficiently implement more traditional strategies for data visualization like small screens and speech/sound output. Their real strength is although more linked to the concepts of glances and low tech outputs that can shuttle simple information to the user in a very fast and effective way.
Simple output components can be combined to implement custom-designed and tailored output primitives. For example a custom shaped led matrix.

\subsubsection{Data processing}
Processing of sensed data can introduce a variety of challenges and demands for compromises. Being TUIs connected devices by nature, two main alternatives are feasible, depending on the nature of data and the processing power available. Data can be either processed on-board, transmitted to an external service for storing and/or processing. An hybrid approach can also be used, where data is pre-elaborated or filtered on board and then transmitted for further processing.
We can then state that the best strategy is influenced by:
\begin{itemize}
    \item characteristics of data stream
    \item complexity of the processing routine
    \item processing power available
    \item network availability and capabilities
\end{itemize}

\subsubsection{User interaction}
In many smart city scenarios, users cannot always devote their full attention to interact with devices. TUIs helps at this regard allowing several interaction strategies that can be less distracting and immersive when the scenario does not require it.
Background and foreground interaction methods are both implementable, as well as multi-modal strategies where users can interact via different channels at the same time. Interaction can also be distributed in the space, involving actions performed on/with different augmented object situated in the space.

\subsubsection{Sharing of artifacts and information}
Technologies like big screens and mobile devices, often used in smart city learning scenarios, present some constraints that are quite rigid and difficult to overcome. For example mobile devices are proven to support cooperative activities and information sharing, but they physically remain strictly personal apparatus that are not comfortably shared even with close persons.
Public displays are instead useful only to present information that is relevant for a defined geographical place, usually where the display itself is located. The nature of information should also be generic enough to not introduce privacy concerns.
With TUIs it is possible to fill the gap between public and personal/private. Devices can be tailored and adapted to support use cases where the users are comfortably sharing objects and information.
The added flexibility allow for several level of experience that can freely space between the \textit{private} and the \textit{public}.


Tools, as well as ICT applications and the Internet itself, may facilitate higher order skills and thus mediate learning, in schools and beyond\cite{kashdan_outdoors_2013}. Indeed, Engestrom\cite{engestrom_non_1991} argues against the separation between schooling and other learning experiences; he refers to it as the encapsulation of school learning. To overcome this encapsulation, he offers emphasis on the role of mediating artifacts in human cognition and learning.
This project achieves overcoming the encapsulation of school learning by using ICT tools to mediate the environment to the students, as well as to expose the products to the community\cite{kashdan_outdoors_2013}.

In \cite{luff_mobility_1998} an analysis of technology-supported coordination mechanisms is provided for several scenarios.
After investigating a urban scenario in London's underground they reported that 

\textit{For example, in distributing technologies around the environment developing support for station supervisors would appear to be a prototypical case of ‘ubiquitous computing’ or
‘augmented reality’. However, although typical developments in these areas aim to support tasks and activities by augmenting everyday artifacts with computational capabilities it is not all that clear which artifacts are most relevant for such enhancement or what capabilities should be augmented}

It is also mentioned that this type of technologies are able to support coordination mechanisms demonstrated to be essential for the scenarios examined, like micro-mobilty, ubiquitous information retrival and glances.

\textit{
In considering the design of the technology however, we have increasingly realised that it may well be a mistake to place access to all necessary resources on the device itself. It is critical for example that the system remains portable, and becomes one of the various tools that station staff carry about with them as part of their normal duties.}

\textit{Nevertheless, it appears from the studies considered in this paper that the micro-mobility of objects may be critical when considering how to support co-present, collaborative activities. To provide for this may require not only both mobile and fixed devices, but quite novel support for mobility that focuses on the moment-to-moment manipulation of objects.}

