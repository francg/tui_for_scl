
%%% citation block from IxD&A
Specific to this project was the implementation of urban informatics techniques and approaches to promote innovative engagement strategies. Architecture and Urban Design students were encouraged to review and appropriate real-time, ubiquitous technology, social media, and mobile devices that were used by urban residents to augment and mediate the physical and digital layers of urban infrastructures. Our study’s experience found that urban informatics provide an innovative opportunity to enrich students’ place of learning within the city\cite{amayocaldwell_urban_2013}

No doubt that among the consequences of such attention there is an acceleration in supporting the integration and embedding of ICT within physical environments to realize what has been defined
the ``everyware''\cite{giffinger_smart_2007}

A new technological infrastructure that could act as a modern ``volumen'' is needed and its goal should be to foster a more intimate contact with the cultural background of cities/territories and to support living experience characterized by an high level of physical involvement. A kind of involvement very different from staying ``glued'' to a laptop screen or to a portable playstation. In this way, perhaps, one will succeed also in avoiding that during the visit of an exhibition the greatest attractions for children would be touch-screens instead of exhibited artifacts.\cite{giovannella_scenarios_2013}


Moreover re-contextualization of pieces of the collections - e.g. archaeological and historical artifacts, historical views provided by photographs, prints, paintings, etc. - may promote: i) the mutual amplification of sense, due to the interplay among re-contextualized "objects" and hosting contexts; ii)
the design of environments more suitable to foster engaging narrative experience of the "places". In fact, the use of content in virtual form allows for its easy manipulation and, thus, to create more engaging inter-actions for specific targets (e.g. to amplify the ludic dimension\cite{greenfield_everyware_2010} of the interaction when children are involved) and stimulate the proactivity of individuals providing them the means to add "sense and meaning"\cite{giovannella_scenarios_2013}

Thanks to such level of integration smart physical environments will allow to play and learn in multi-user modality, being people physically or virtually delocalized (i.e. active in other smart physical environments, or in other virtual environments).\cite{giovannella_scenarios_2013}

Jamieson et al. \cite{jamieson_place_2000} stress that previous research in the area of learning environments has focused on students’ learning experience and the teaching approach. The concern has not been directed towards the relationship between the design of physical environments and their creation of places of learning. Jamieson et al. \cite{jamieson_place_2000} therefore argue that there is a strong connection between the quality of the learning environments and the experience of place affecting learning outcomes and student success. Although on-campus teaching and learning environments, which include both digital and physical environments, occur in physical place, the incorporation of ICTs has affected the teacher / student relationships to it \cite{jamieson_place_2000}. 

As educators we need to provide environments that are comfortable, where students can take risks and challenge their preconceptions while learning new modes of critical thinking \cite{kolb_learning_2000}. It is through the exposure of real world urban sites and issues that this study has focused towards shifting the space of learning to a place of learning in the smart city context.\cite{amayocaldwell_urban_2013}

The use of technology and its ubiquitous nature, has the potential to encourage a wider range of participation through the sharing of information \cite{fredericks_augmenting_2013} \cite{houghton_appropriating_2013}.

The primary characteristics of GRT are that covert tactics are developed for attracting and engaging with research participants typically when aquiring data from the public is usually challenging using conventional means.\cite{amayocaldwell_urban_2013}

The GRT methods can employ physical artefacts containing digital links to online data collecting resources such as an online poll or survey that can also be distributed through social networks.\cite{amayocaldwell_urban_2013}

Tim Campbell, author of Beyond Smart Cities\cite{campbell_smart_2013}, states that learning occurs in the minds of people who are interested in taking action and who care about the cities where they live. Campbell claims that the most active learning cities create methods to facilitate the creation, sharing, and acquisition of new ideas to assist in solving local urban problems. \cite{amayocaldwell_urban_2013}

Mobile learning is “any sort of learning that happens when the learner is not at a fixed, predetermined location, or learning that happens when the learner takes advantage of the learning opportunities offered by mobile technologies” \cite{omalley_guidelines_2005}. Ubiquitous learning on the other hand means something more than mobility, a ubiquitous learning environment is any setting in which students can become totally immersed in the learning process \cite{syvanen_supporting_2005}; it is a setting of pervasive learning that is happening all around the student but the student may not even be conscious of the learning process. In extension, smart city learning leverages the infrastructure and services that modern smart cities offer in order to transform them into open learning environments.\cite{christopoulou_learning_2013}

As younger generations are “born” familiar with those systems there is an increasing benefit in adopting such technologies, as they facilitate students to continue their learning process outside classrooms, when and where they desire, through exploration and interaction. Therefore, an increasing interest in physical spaces, in the role of technology in city-wide environments and in passage of learning activities outside the classroom is revealed.\cite{christopoulou_learning_2013}

%%%%%% 