% This is "sig-alternate.tex" V2.1 April 2013
% This file should be compiled with V2.5 of "sig-alternate.cls" May 2012
%
% This example file demonstrates the use of the 'sig-alternate.cls'
% V2.5 LaTeX2e document class file. It is for those submitting
% articles to ACM Conference Proceedings WHO DO NOT WISH TO
% STRICTLY ADHERE TO THE SIGS (PUBS-BOARD-ENDORSED) STYLE.
% The 'sig-alternate.cls' file will produce a similar-looking,
% albeit, 'tighter' paper resulting in, invariably, fewer pages.
%
% ----------------------------------------------------------------------------------------------------------------
% This .tex file (and associated .cls V2.5) produces:
%       1) The Permission Statement
%       2) The Conference (location) Info information
%       3) The Copyright Line with ACM data
%       4) NO page numbers
%
% as against the acm_proc_article-sp.cls file which
% DOES NOT produce 1) thru' 3) above.
%
% Using 'sig-alternate.cls' you have control, however, from within
% the source .tex file, over both the CopyrightYear
% (defaulted to 200X) and the ACM Copyright Data
% (defaulted to X-XXXXX-XX-X/XX/XX).
% e.g.
% \CopyrightYear{2007} will cause 2007 to appear in the copyright line.
% \crdata{0-12345-67-8/90/12} will cause 0-12345-67-8/90/12 to appear in the copyright line.
%
% ---------------------------------------------------------------------------------------------------------------
% This .tex source is an example which *does* use
% the .bib file (from which the .bbl file % is produced).
% REMEMBER HOWEVER: After having produced the .bbl file,
% and prior to final submission, you *NEED* to 'insert'
% your .bbl file into your source .tex file so as to provide
% ONE 'self-contained' source file.
%
% ================= IF YOU HAVE QUESTIONS =======================
% Questions regarding the SIGS styles, SIGS policies and
% procedures, Conferences etc. should be sent to
% Adrienne Griscti (griscti@acm.org)
%
% Technical questions _only_ to
% Gerald Murray (murray@hq.acm.org)
% ===============================================================
%
% For tracking purposes - this is V2.0 - May 2012

\documentclass{sig-alternate-05-2015}

\usepackage[table,xcdraw]{xcolor}

\begin{document}

% Copyright
% \setcopyright{acmcopyright}
%\setcopyright{acmlicensed}
%\setcopyright{rightsretained}
%\setcopyright{usgov}
%\setcopyright{usgovmixed}
%\setcopyright{cagov}
%\setcopyright{cagovmixed}


% DOI
% \doi{10.475/123_4}

% ISBN
% \isbn{123-4567-24-567/08/06}

%Conference
\conferenceinfo{AVI '16}{June 7--10, 2016, Bari, Italy}

% \acmPrice{\$15.00}

%
% --- Author Metadata here ---
% \conferenceinfo{WOODSTOCK}{'97 El Paso, Texas USA}
%\CopyrightYear{2007} % Allows default copyright year (20XX) to be over-ridden - IF NEED BE.
%\crdata{0-12345-67-8/90/01}  % Allows default copyright data (0-89791-88-6/97/05) to be over-ridden - IF NEED BE.
% --- End of Author Metadata ---

\title{Exploring Tangible Interfaces for Smart City Learning}
% \subtitle{[Extended Abstract]
%
% You need the command \numberofauthors to handle the 'placement
% and alignment' of the authors beneath the title.
%
% For aesthetic reasons, we recommend 'three authors at a time'
% i.e. three 'name/affiliation blocks' be placed beneath the title.
%
% NOTE: You are NOT restricted in how many 'rows' of
% "name/affiliations" may appear. We just ask that you restrict
% the number of 'columns' to three.
%
% Because of the available 'opening page real-estate'
% we ask you to refrain from putting more than six authors
% (two rows with three columns) beneath the article title.
% More than six makes the first-page appear very cluttered indeed.
%
% Use the \alignauthor commands to handle the names
% and affiliations for an 'aesthetic maximum' of six authors.
% Add names, affiliations, addresses for
% the seventh etc. author(s) as the argument for the
% \additionalauthors command.
% These 'additional authors' will be output/set for you
% without further effort on your part as the last section in
% the body of your article BEFORE References or any Appendices.

\numberofauthors{3} %  in this sample file, there are a *total*
% of EIGHT authors. SIX appear on the 'first-page' (for formatting
% reasons) and the remaining two appear in the \additionalauthors section.
%
\author{
% You can go ahead and credit any number of authors here,
% e.g. one 'row of three' or two rows (consisting of one row of three
% and a second row of one, two or three).
%
% The command \alignauthor (no curly braces needed) should
% precede each author name, affiliation/snail-mail address and
% e-mail address. Additionally, tag each line of
% affiliation/address with \affaddr, and tag the
% e-mail address with \email.
%
% 1st. author
\alignauthor
Francesco Gianni\\
       \affaddr{Norwegian Univesity of Science and Technology}\\
       \affaddr{Department of Computer and Information Science}\\
       \affaddr{Trondheim, Norway}\\
       \email{francesco.gianni@idi.ntnu.no}
% 2nd. author
\alignauthor
Simone Mora\\
       \affaddr{Norwegian Univesity of Science and Technology}\\
       \affaddr{Department of Computer and Information Science}\\
       \affaddr{Trondheim, Norway}\\
       \email{simonem@ntnu.no}
% 3rd. author
\alignauthor Monica Divitini\\
       \affaddr{Norwegian Univesity of Science and Technology}\\
       \affaddr{Department of Computer and Information Science}\\
       \affaddr{Trondheim, Norway}\\
       \email{monica.divitini@idi.ntnu.no}
}
% There's nothing stopping you putting the seventh, eighth, etc.
% author on the opening page (as the 'third row') but we ask,
% for aesthetic reasons that you place these 'additional authors'
% in the \additional authors block, viz.

% Just remember to make sure that the TOTAL number of authors
% is the number that will appear on the first page PLUS the
% number that will appear in the \additionalauthors section.

\maketitle
\begin{abstract}
Pervasive information visualisation is a fundamental tool to support learning in smart cities, for example to promote sustainable behaviors and social interaction. In this setting, information supporting learning goals is usually provided via screen-based interfaces such as public large displays and smart-phones. Those interfaces pose several usability issues and ignore the space of opportunities offered by research in tangible and embodied interaction.
Building on a review of existing works, we identify the main limitations of traditional approaches based on large displays and smart-phones, first from a technological point of view then connecting to implications for user interaction and experience.
Opportunities and challenges are then compared to the peculiarities of TUIs that can be useful and effective in smart city learning scenarios.
%Building on a review of existing works, we identify opportunities and challenges related to the development TUIs for smart city learning.
We explore new design opportunities allowed by TUIs in smart city learning scenarios of interest. We propose the use of \emph{Tiles}, an experimental prototyping platform, to quickly implement scenarios in working prototypes.
\emph{Tiles} allow to digitalize and augment regular objects, collect sensor data and handle user notification and interaction in several ways using onboard I/O components.
The architecture behind the electronic hardware is cloud-based. \emph{Tiles} functions are wrapped behind a simple, language-agnostic API, that allow both developers and designers to take advantage of the capabilities offered.
%We explore those design opportunities by providing smart city learning scenarios and we propose the use of \emph{Tiles}, an experimental prototyping platform, to quickly implement scenarios in working prototypes.  

\end{abstract}


%
% The code below should be generated by the tool at
% http://dl.acm.org/ccs.cfm
% Please copy and paste the code instead of the example below. 
%
\begin{CCSXML}
<ccs2012>
 <concept>
  <concept_id>10010520.10010553.10010562</concept_id>
  <concept_desc>Computer systems organization~Embedded systems</concept_desc>
  <concept_significance>500</concept_significance>
 </concept>
 <concept>
  <concept_id>10010520.10010575.10010755</concept_id>
  <concept_desc>Computer systems organization~Redundancy</concept_desc>
  <concept_significance>300</concept_significance>
 </concept>
 <concept>
  <concept_id>10010520.10010553.10010554</concept_id>
  <concept_desc>Computer systems organization~Robotics</concept_desc>
  <concept_significance>100</concept_significance>
 </concept>
 <concept>
  <concept_id>10003033.10003083.10003095</concept_id>
  <concept_desc>Networks~Network reliability</concept_desc>
  <concept_significance>100</concept_significance>
 </concept>
</ccs2012>  
\end{CCSXML}

\ccsdesc[500]{Computer systems organization~Embedded systems}
\ccsdesc[300]{Computer systems organization~Redundancy}
\ccsdesc{Computer systems organization~Robotics}
\ccsdesc[100]{Networks~Network reliability}


%
% End generated code
%

%
%  Use this command to print the description
%
\printccsdesc

% We no longer use \terms command
%\terms{Theory}

\keywords{ACM proceedings; \LaTeX; text tagging}


\section{Introduction}
Research and innovation in the smart city is becoming a critical field for our society. Half of the world population is already living in cities, research in smart cities can have a big impact on a considerable fraction of world's population.
Studies demonstrate that social connections in cities stimulates creativity and improves work quality\cite{florida_cities_2005}. This is only one of the reasons why the percentage of people living in urban environments is growing.

Cities are important research playgrounds for substantial improvements in safety and quality of life.
Flow of information and people are easily observable in a highly connected urban environment, at the same time these information affect a high percentage of the humanity. A critical scenario emerges, where decision makers, citizens and researchers can shape the urban environment that influence our society.

It is also important to support the learning process within the urban communities: \textit{lifelong learning} can be used to promote sustainable behaviors and knowledge building in the city, with the ultimate goal of increasing community awareness and participatory creation of the urban space and thus increase quality of citizens' life.

Smart cities present, by definition, a strong technological component.
In Technology Enhanced Learning (TEL), the role of technology is to direct, foster thinking and facilitate the acquisition of higher order skills\cite{goodyear_technologyenhanced_2010}.
Current research applied to learning in the cities seem to focus on two main technological means for learning contents: situated large displays and mobile devices, intended as tablets and smart-phones\cite{luff_mobility_1998}.
% add more ref

Yet the affordances of digital screens are not always suited for informing people in urban settings \cite{koeman}. While large displays suffer from issues known as "display blindess" \cite{}, and in engaging users to interact with them \cite{}; mobile displays (e.g. smartphones, tablets) can be an obstacle to social interaction and are not suitable or safe to be used while in mobility for example walking or biking \cite{}. 

We claim there is a space of opportunities for smart city learning applications in adopting novel ubiquitous computing approaches like tangible interfaces and augmented objects. These technologies have already be found effective in supporting learning\cite{stanton_classroom_2001}, but their applications were mainly oriented to support learning as it happen in conventional schools and classrooms. The principal advantages in adopting these types of interfaces are (i) to enable the creation of rich and unobtrusive user experiences, (ii) to extend the type of data that can be captured to be used as learning content, including sensor data from the environment and from citizens' whereabouts. Therefore sensor-based tangible interfaces could complement traditional approaches based on large screens and smartphones, especially when the learning environment can be as wide and heterogeneous as a city. Todays' increasingly adoption of sensors and IoT technologies are acting as enabling factors for the development of such interfaces; yet whether a number of studies have reported design guidelines for urban screens \cite{}, there a lack of guidelines to help the design of different type of interfaces.

Toolkits play a role for facilitating the design and building prototypes of such applications.

In this paper we...

Our approach is...



to research new forms of utilizing and collecting data in the city and at the same time experimenting with novel strategies of interaction between users and technology.
The potential of custom-designed hardware can open new possibilities in terms of communication efficiency, methods of interaction with the user and sensor data collection.
Glances, gesture based input and micro interactions are some of the approaches possible with the use of latest toolkits for IoT.
%% ref toolkit

Traditional technology is otherwise a limiting factor: mobile devices and large screens support a very strict and confined set of interaction strategies. It's often not possible to tailor the user experience to properly fit the specific scenario because technology is too limiting.
Our goal is to design aiming at the best possible strategy for the users, building the technology around this process and avoiding the constraints typically introduced by more general-purpose hardware/software combinations.



\section{Tangible interfaces for smart city learning}
\subsection{Connection with smart cities}

Some of the activities that are needed and connected to the smart city learning scenarios involves:

\begin{itemize}
\item data collection
\item data visualization
\item data processing
\item user interaction
\item sharing and cooperation of both electronics and informations
\end{itemize}

We identified how these activities can be supported with tangible interfaces.

\subsubsection{Data collection}
Sensor data can be captured embedding proper electronics in augmented objects or electronics toolkits.
Some of this \textit{sensing} capabilities are necessary to actually allow augmented user interactions (e.g. accelerometers, gyro).
Despite this, raw data sensed can be manipulated and processed also for other purposed more connected to the ambient sensing than remaining limited to only sense user (rich) interaction.
The opportunity to flexibly and quickly prototype custom devices, allow to embed with a relatively low cost several different sensors.
The opportunity to situate devices both on the environment and physically on the users widens the domain where data collection take place.

\subsubsection{Data visualization}
Citizens living everyday in the city can be subjected to common life situations where the most suitable strategies of consuming information can be fairly different.
TUIs can efficiently implement more traditional strategies for data visualization like small screens and speech/sound output. Their real strength is although more linked to the concepts of glances and low tech outputs that can shuttle simple information to the user in a very fast and effective way.
Simple output components can be combined to implement custom-designed and tailored output primitives. For example a custom shaped led matrix.

\subsubsection{Data processing}
Processing of sensed data can introduce a variety of challenges and demands for compromises. Being TUIs connected devices by nature, two main alternatives are feasible, depending on the nature of data and the processing power available. Data can be either processed on-board, transmitted to an external service for storing and/or processing. An hybrid approach can also be used, where data is pre-elaborated or filtered on board and then transmitted for further processing.
We can then state that the best strategy is influenced by:
\begin{itemize}
    \item characteristics of data stream
    \item complexity of the processing routine
    \item processing power available
    \item network availability and capabilities
\end{itemize}

\subsubsection{User interaction}
In many smart city scenarios, users cannot always devote their full attention to interact with devices. TUIs helps at this regard allowing several interaction strategies that can be less distracting and immersive when the scenario does not require it.
Background and foreground interaction methods are both implementable, as well as multi-modal strategies where users can interact via different channels at the same time. Interaction can also be distributed in the space, involving actions performed on/with different augmented object situated in the space.

\subsubsection{Sharing of artifacts and information}
Technologies like big screens and mobile devices, often used in smart city learning scenarios, present some constraints that are quite rigid and difficult to overcome. For example mobile devices are proven to support cooperative activities and information sharing, but they physically remain strictly personal apparatus that are not comfortably shared even with close persons.
Public displays are instead useful only to present information that is relevant for a defined geographical place, usually where the display itself is located. The nature of information should also be generic enough to not introduce privacy concerns.
With TUIs it is possible to fill the gap between public and personal/private. Devices can be tailored and adapted to support use cases where the users are comfortably sharing objects and information.
The added flexibility allow for several level of experience that can freely space between the \textit{private} and the \textit{public}.


Tools, as well as ICT applications and the Internet itself, may facilitate higher order skills and thus mediate learning, in schools and beyond\cite{kashdan_outdoors_2013}. Indeed, Engestrom\cite{engestrom_non_1991} argues against the separation between schooling and other learning experiences; he refers to it as the encapsulation of school learning. To overcome this encapsulation, he offers emphasis on the role of mediating artifacts in human cognition and learning.
This project achieves overcoming the encapsulation of school learning by using ICT tools to mediate the environment to the students, as well as to expose the products to the community\cite{kashdan_outdoors_2013}.

In \cite{luff_mobility_1998} an analysis of technology-supported coordination mechanisms is provided for several scenarios.
After investigating a urban scenario in London's underground they reported that 

\textit{For example, in distributing technologies around the environment developing support for station supervisors would appear to be a prototypical case of ‘ubiquitous computing’ or
‘augmented reality’. However, although typical developments in these areas aim to support tasks and activities by augmenting everyday artifacts with computational capabilities it is not all that clear which artifacts are most relevant for such enhancement or what capabilities should be augmented}

It is also mentioned that this type of technologies are able to support coordination mechanisms demonstrated to be essential for the scenarios examined, like micro-mobilty, ubiquitous information retrival and glances.

\textit{
In considering the design of the technology however, we have increasingly realised that it may well be a mistake to place access to all necessary resources on the device itself. It is critical for example that the system remains portable, and becomes one of the various tools that station staff carry about with them as part of their normal duties.}

\textit{Nevertheless, it appears from the studies considered in this paper that the micro-mobility of objects may be critical when considering how to support co-present, collaborative activities. To provide for this may require not only both mobile and fixed devices, but quite novel support for mobility that focuses on the moment-to-moment manipulation of objects.}



\section{Tangible user interfaces as a tool}
% Research and innovation in the smart city is becoming a critical field for our society. Half of the world population is already living in cities, research in smart cities can have a big impact on a considerable fraction of world's population.
Studies demonstrate that social connections in cities stimulates creativity and improves work quality\cite{florida_cities_2005}. This is only one of the reasons why the percentage of people living in urban environments is growing.

Cities are important research playgrounds for substantial improvements in safety and quality of life.
Flow of information and people are easily observable in a highly connected urban environment, at the same time these information affect a high percentage of the humanity. A critical scenario emerges, where decision makers, citizens and researchers can shape the urban environment that influence our society.

It is also important to support the learning process within the urban communities: \textit{lifelong learning} can be used to promote sustainable behaviors and knowledge building in the city, with the ultimate goal of increasing community awareness and participatory creation of the urban space and thus increase quality of citizens' life.

Smart cities present, by definition, a strong technological component.
In Technology Enhanced Learning (TEL), the role of technology is to direct, foster thinking and facilitate the acquisition of higher order skills\cite{goodyear_technologyenhanced_2010}.
Current research applied to learning in the cities seem to focus on two main technological means for learning contents: situated large displays and mobile devices, intended as tablets and smart-phones\cite{luff_mobility_1998}.
% add more ref

Yet the affordances of digital screens are not always suited for informing people in urban settings \cite{koeman}. While large displays suffer from issues known as "display blindess" \cite{}, and in engaging users to interact with them \cite{}; mobile displays (e.g. smartphones, tablets) can be an obstacle to social interaction and are not suitable or safe to be used while in mobility for example walking or biking \cite{}. 

We claim there is a space of opportunities for smart city learning applications in adopting novel ubiquitous computing approaches like tangible interfaces and augmented objects. These technologies have already be found effective in supporting learning\cite{stanton_classroom_2001}, but their applications were mainly oriented to support learning as it happen in conventional schools and classrooms. The principal advantages in adopting these types of interfaces are (i) to enable the creation of rich and unobtrusive user experiences, (ii) to extend the type of data that can be captured to be used as learning content, including sensor data from the environment and from citizens' whereabouts. Therefore sensor-based tangible interfaces could complement traditional approaches based on large screens and smartphones, especially when the learning environment can be as wide and heterogeneous as a city. Todays' increasingly adoption of sensors and IoT technologies are acting as enabling factors for the development of such interfaces; yet whether a number of studies have reported design guidelines for urban screens \cite{}, there a lack of guidelines to help the design of different type of interfaces.

Toolkits play a role for facilitating the design and building prototypes of such applications.

In this paper we...

Our approach is...



to research new forms of utilizing and collecting data in the city and at the same time experimenting with novel strategies of interaction between users and technology.
The potential of custom-designed hardware can open new possibilities in terms of communication efficiency, methods of interaction with the user and sensor data collection.
Glances, gesture based input and micro interactions are some of the approaches possible with the use of latest toolkits for IoT.
%% ref toolkit

Traditional technology is otherwise a limiting factor: mobile devices and large screens support a very strict and confined set of interaction strategies. It's often not possible to tailor the user experience to properly fit the specific scenario because technology is too limiting.
Our goal is to design aiming at the best possible strategy for the users, building the technology around this process and avoiding the constraints typically introduced by more general-purpose hardware/software combinations.



\section{Design implications}
% Research and innovation in the smart city is becoming a critical field for our society. Half of the world population is already living in cities, research in smart cities can have a big impact on a considerable fraction of world's population.
Studies demonstrate that social connections in cities stimulates creativity and improves work quality\cite{florida_cities_2005}. This is only one of the reasons why the percentage of people living in urban environments is growing.

Cities are important research playgrounds for substantial improvements in safety and quality of life.
Flow of information and people are easily observable in a highly connected urban environment, at the same time these information affect a high percentage of the humanity. A critical scenario emerges, where decision makers, citizens and researchers can shape the urban environment that influence our society.

It is also important to support the learning process within the urban communities: \textit{lifelong learning} can be used to promote sustainable behaviors and knowledge building in the city, with the ultimate goal of increasing community awareness and participatory creation of the urban space and thus increase quality of citizens' life.

Smart cities present, by definition, a strong technological component.
In Technology Enhanced Learning (TEL), the role of technology is to direct, foster thinking and facilitate the acquisition of higher order skills\cite{goodyear_technologyenhanced_2010}.
Current research applied to learning in the cities seem to focus on two main technological means for learning contents: situated large displays and mobile devices, intended as tablets and smart-phones\cite{luff_mobility_1998}.
% add more ref

Yet the affordances of digital screens are not always suited for informing people in urban settings \cite{koeman}. While large displays suffer from issues known as "display blindess" \cite{}, and in engaging users to interact with them \cite{}; mobile displays (e.g. smartphones, tablets) can be an obstacle to social interaction and are not suitable or safe to be used while in mobility for example walking or biking \cite{}. 

We claim there is a space of opportunities for smart city learning applications in adopting novel ubiquitous computing approaches like tangible interfaces and augmented objects. These technologies have already be found effective in supporting learning\cite{stanton_classroom_2001}, but their applications were mainly oriented to support learning as it happen in conventional schools and classrooms. The principal advantages in adopting these types of interfaces are (i) to enable the creation of rich and unobtrusive user experiences, (ii) to extend the type of data that can be captured to be used as learning content, including sensor data from the environment and from citizens' whereabouts. Therefore sensor-based tangible interfaces could complement traditional approaches based on large screens and smartphones, especially when the learning environment can be as wide and heterogeneous as a city. Todays' increasingly adoption of sensors and IoT technologies are acting as enabling factors for the development of such interfaces; yet whether a number of studies have reported design guidelines for urban screens \cite{}, there a lack of guidelines to help the design of different type of interfaces.

Toolkits play a role for facilitating the design and building prototypes of such applications.

In this paper we...

Our approach is...



to research new forms of utilizing and collecting data in the city and at the same time experimenting with novel strategies of interaction between users and technology.
The potential of custom-designed hardware can open new possibilities in terms of communication efficiency, methods of interaction with the user and sensor data collection.
Glances, gesture based input and micro interactions are some of the approaches possible with the use of latest toolkits for IoT.
%% ref toolkit

Traditional technology is otherwise a limiting factor: mobile devices and large screens support a very strict and confined set of interaction strategies. It's often not possible to tailor the user experience to properly fit the specific scenario because technology is too limiting.
Our goal is to design aiming at the best possible strategy for the users, building the technology around this process and avoiding the constraints typically introduced by more general-purpose hardware/software combinations.



\section{Conclusion}
% Research and innovation in the smart city is becoming a critical field for our society. Half of the world population is already living in cities, research in smart cities can have a big impact on a considerable fraction of world's population.
Studies demonstrate that social connections in cities stimulates creativity and improves work quality\cite{florida_cities_2005}. This is only one of the reasons why the percentage of people living in urban environments is growing.

Cities are important research playgrounds for substantial improvements in safety and quality of life.
Flow of information and people are easily observable in a highly connected urban environment, at the same time these information affect a high percentage of the humanity. A critical scenario emerges, where decision makers, citizens and researchers can shape the urban environment that influence our society.

It is also important to support the learning process within the urban communities: \textit{lifelong learning} can be used to promote sustainable behaviors and knowledge building in the city, with the ultimate goal of increasing community awareness and participatory creation of the urban space and thus increase quality of citizens' life.

Smart cities present, by definition, a strong technological component.
In Technology Enhanced Learning (TEL), the role of technology is to direct, foster thinking and facilitate the acquisition of higher order skills\cite{goodyear_technologyenhanced_2010}.
Current research applied to learning in the cities seem to focus on two main technological means for learning contents: situated large displays and mobile devices, intended as tablets and smart-phones\cite{luff_mobility_1998}.
% add more ref

Yet the affordances of digital screens are not always suited for informing people in urban settings \cite{koeman}. While large displays suffer from issues known as "display blindess" \cite{}, and in engaging users to interact with them \cite{}; mobile displays (e.g. smartphones, tablets) can be an obstacle to social interaction and are not suitable or safe to be used while in mobility for example walking or biking \cite{}. 

We claim there is a space of opportunities for smart city learning applications in adopting novel ubiquitous computing approaches like tangible interfaces and augmented objects. These technologies have already be found effective in supporting learning\cite{stanton_classroom_2001}, but their applications were mainly oriented to support learning as it happen in conventional schools and classrooms. The principal advantages in adopting these types of interfaces are (i) to enable the creation of rich and unobtrusive user experiences, (ii) to extend the type of data that can be captured to be used as learning content, including sensor data from the environment and from citizens' whereabouts. Therefore sensor-based tangible interfaces could complement traditional approaches based on large screens and smartphones, especially when the learning environment can be as wide and heterogeneous as a city. Todays' increasingly adoption of sensors and IoT technologies are acting as enabling factors for the development of such interfaces; yet whether a number of studies have reported design guidelines for urban screens \cite{}, there a lack of guidelines to help the design of different type of interfaces.

Toolkits play a role for facilitating the design and building prototypes of such applications.

In this paper we...

Our approach is...



to research new forms of utilizing and collecting data in the city and at the same time experimenting with novel strategies of interaction between users and technology.
The potential of custom-designed hardware can open new possibilities in terms of communication efficiency, methods of interaction with the user and sensor data collection.
Glances, gesture based input and micro interactions are some of the approaches possible with the use of latest toolkits for IoT.
%% ref toolkit

Traditional technology is otherwise a limiting factor: mobile devices and large screens support a very strict and confined set of interaction strategies. It's often not possible to tailor the user experience to properly fit the specific scenario because technology is too limiting.
Our goal is to design aiming at the best possible strategy for the users, building the technology around this process and avoiding the constraints typically introduced by more general-purpose hardware/software combinations.




%
% The following two commands are all you need in the
% initial runs of your .tex file to
% produce the bibliography for the citations in your paper.
\bibliographystyle{abbrv}
\bibliography{bib_avi}  % sigproc.bib is the name of the Bibliography in this case
% You must have a proper ".bib" file
%  and remember to run:
% latex bibtex latex latex
% to resolve all references
%
% ACM needs 'a single self-contained file'!

\end{document}
